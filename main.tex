\documentclass[a4paper]{article}

\usepackage[english]{babel}
\usepackage[utf8]{inputenc}
\usepackage{graphicx}
\usepackage{epsfig}
\usepackage{amsmath}
\usepackage{graphicx}
\usepackage[colorinlistoftodos]{todonotes}
\usepackage{a4}
\usepackage{caption}

%\usepackage{amssymb}
\usepackage{color}
\usepackage{lineno}
\usepackage{ulem}
\usepackage{enumerate}
\usepackage{comment}

\usepackage[left=2.5cm,right=2cm,top=2.5cm,bottom=2.cm]{geometry} 

%% for long url reference
\usepackage{hyperref}

\usepackage{xcolor}

\hypersetup{colorlinks=false,linkbordercolor=red,linkcolor=green,pdfborderstyle={/S/U/W 1}}

\usepackage{url}
\makeatletter
\def\url@mystyle{%
  \@ifundefined{selectfont}{\def\UrlFont{\sf}}{\def\UrlFont{\small\ttfamily}}}
\makeatother
\urlstyle{my}



\renewcommand{\thefootnote}{\alph{footnote}}
\renewcommand{\topfraction}{.99}
\renewcommand{\bottomfraction}{.99}

\title{Total Hadronic  (K$^+$, Ar) Cross Section for Run-II}

%%%%%%%%%%%%%%%%%%%%%%%%%%%%%%%%%
\begin{document}
%%%%%%%%%%%%%%%%%%%%%%%%%%%%%%%%%
\def\Journal#1#2#3#4{{#1} {\bf #2}, #3 (#4)}
\def\etal{{\it et\ al.}}
\def\numunue{\nu_\mu\rightarrow\nu_e}
\def\numunutau{\nu_\mu\rightarrow\nu_\tau}
\def\nuebar{\bar\nu_e}
\def\nue{\nu_e}
\def\nutau{\nu_\tau}
\def\numubar{\bar\nu_\mu}
\def\numu{\nu_\mu}
\def\ra{\rightarrow}
\def\numubarnuebar{\bar\nu_\mu\rightarrow\bar\nu_e}
\def\nuebarnumubar{\bar\nu_e\rightarrow\bar\nu_\mu}
\def\osc{\rightsquigarrow}
\def\inteni{{\cal I}_{pot}}
\def\fmerit{{\cal F}}
%%%%%%%%%%%%%%%%%%%%%%%%%%%%%%%%%
\begin{flushright}
{\tt version -1.0}\\ 
\today
\end{flushright}
\vspace*{0.6cm}
%%%%%%%%%%%%%%%%%%%%%%%%%%%%%%%%%
%\linenumbers
%%%%%%%%%%%%%%%%%%%%%%%%%%%%%%%%%
\begin{center}
{\Large \bf Total Hadronic  (K$^+$, Ar) Cross Section for Run-II} 
\vspace*{1.6cm}
\setcounter{footnote}{0}  
\def\A{\kern+.6ex\lower.42ex\hbox{$\scriptstyle \iota$}\kern-1.20ex a}
\def\E{\kern+.5ex\lower.42ex\hbox{$\scriptstyle \iota$}\kern-1.10ex e}
\small
\newcommand{\Aname}[2]{#1}
\def\titlefoot#1{\vspace{-0.3cm}\begin{center}{\bf #1}\end{center}}

Authors: %Roberto Acciarri, Jonathan Asaadi\footnote{Much of the writing was done by this author, please blame him for misrepresentations and mistakes. Credit and successes belong to the others}, Flor de Maria Blaszczyk, \\ 
%Flavio Cavanna, Animesh Chatterjee, Tapasi Ghosh, Elena Gramellini,\\ 
% Johnny Ho, Pawel Kryczynski, Celio Moura, Irene Nutini, \\
%Greg Pulliam, Jennifer Raaf, Jason St.John, Brandon Soubasis, Tingjun Yang. \\

\end{center}
\vspace*{1cm}


%%%%%%%%%%%%%%%%%%%%%%%%%%%%%%%%%
%% ABSTRACT
%%%%%%%%%%%%%%%%%%%%%%%%%%%%%%%%%
%\newpage
\begin{abstract}

We present the study of the total hadronic positive kaon-argon nucleus interaction cross section performed at the LArIAT experiment. The LArIAT beamline instruments are used to identify candidate kaons and measure their momentum prior to entering the liquid argon time projection chamber (LArTPC). We then use the calorimetric reconstruction of the LArTPC to perform the measurement of the total differential cross section for interacting kaons ($K$) on an argon (Ar) nucleus. The $K^+$-Ar total interaction cross section has never been measured before and it is fundamental to shed light on light meson interactions in nuclei. Additionally, this measurement provides a key input to proton decay studies in future Liquid Argon Time Projection Chamber (LArTPC) experiments such as DUNE.

\end{abstract} 

%%%%%%%%%%%%%%%%%%%%%%%%%%%%%%%%%
%% Table of content
%%%%%%%%%%%%%%%%%%%%%%%%%%%%%%%%%
\tableofcontents


%%%%%%%%%%%%%%%%%%%%%%%%%%%%%%%%%
%% SECTION 1: Introduction
%%%%%%%%%%%%%%%%%%%%%%%%%%%%%%%%%
\newpage
\section{Introduction}\label{sec:Introduction}
In this note, we present the analysis for the total inclusive positively charged kaon - argon ($K^{+}$,Ar) interaction cross section for LArIAT data collected over Run-II. The note is divided into six sections. Section \ref{sec:Introduction} gives an introduction to $K$ interaction cross section measurement and its general physics context. Section \ref{sec:kaonAnalysis} provides an overview of the inclusive $K$ cross section analysis and the ``thin slice'' method. Section \ref{sec:DataSamples} lays out the data and Monte Carlo samples used in this analysis.  Section \ref{sec:BeamlineSelection} describe the LArIAT beamline and its role in the selection of the kaon-candidate events. Section \ref{sec:TPC} describes the treatment of the kaon candidates within the TPC.
Section \ref{sec:Systematics} describes the studies of the systematic uncertainties associated with the measurement. Finally, Section \ref{sec:Results} gives the results for the analysis. 

%%%%%%%%%%%%%%%%%%%%%%%%%%%%%%%%%%%%%%%%%%%%%%%%%%%%%%%%%%%%%%%%%%%%%%%%%%%%%%%%%
\subsection{Kaon interaction cross section}\label{sec:KCrossSection}
%%%%%%%%%%%%%%%%%%%%%%%%%%%%%%%%%%%%%%%%%%%%%%%%%%%%%%%%%%%%%%%%%%%%%%%%%%%%%%%%%
The interaction of a mildly relativistic charged kaon with an argon nucleus is determined largely by the strong force. The total K$^{+}$-Ar interaction cross section is defined as the one related to the single (hadronic) process driven only by the strong interaction.
In this case, ``total" indicates all strong interactions regardless of the final state. This condition purposefully includes both elastic and inelastic (reaction) channels. Indeed, the total cross section section can be then decomposed into
$$\sigma_{Tot} = \sigma_{Elastic}+ \sigma_{Reaction}.$$

For this analysis, kaons are selected from the LArIAT beamline in the momentum range between \textcolor{red}{500} MeV/c and \textcolor{red}{1000} MeV/c (see Fig \ref{fig:TOFK}).

\begin{figure}[hpbt]
\centering
\includegraphics[width=5in]{images/Lariat/KaonTOF}
\caption{Time of flight versus momentum distributions as produced by the LArIAT TOF and Wire Chambers systems. The Kaon population lies between the proton and the muon/pion populations, allowing PID of Kaons in the beam line.  }
\label{fig:TOFK}
\end{figure}

In this energy range, the relevant K-Nucleon interactions are according to \cite{fesbach1992theoretical}:
\begin{equation}
K^{+} + N \rightarrow K^{+} + N\textit{ (elastic)}
\end{equation}
\begin{equation}K^{+} + n \rightarrow K^{0} + p\textit{ (elastic)}\end{equation}
\begin{equation}K^{+} + N \rightarrow K + N + \pi \textit{ (inelastic)}\end{equation}
\begin{equation}K^{+} + N \rightarrow K^{*} + N\textit{ (inelastic)}.\end{equation}



\subsubsection{K$^{+}$Ar Cross section in the Context of Light Mesons Interaction with Nuclei}
\label{sec:theoryStrangeMeson}
The intrinsic value of this measurement is that kaon interactions complement the measurements of $\pi$ interactions as a probe of  hadron interaction inside the nucleus in the strange sector.  
\textcolor{red}{CHIEDI REFERENZE A FLAVIO}

\subsubsection{K$^{+}$Ar Cross section in the Context of Nucleon Decay Searches}
\label{sec:theoryPDK}
Baryon number is accidentally conserved in the Standard Model. Even though no baryon number violation has been experimentally observed thus far, no underlying symmetry in line with the Noether paradigm \cite{Noether1971} explains its conservation. Almost all Grand Unified Theories predict at some level baryon number violation in the form of nucleon decay on long time-scales.  Given the impossibility to reach grand unification energy scales with collider experiments ($\sqrt{s} > 10^{15}$ GeV),  an indirect proof of GUT is needed. The experimental observation of nucleon decay may be the only viable way to explore these theories and it is therefore a subject of great interest \cite{Adams:LBNE}. %Both experiments and theory indeed suggest the energy scale for convergence of the running coupling constants of the Standard Model to be over $10^{15}$ GeV. This energy scale seems impossible to access by any foreseeable accelerator experiment, leaving baryon number violation  to be the only testable process. 

If nucleon decay was experimentally found, additional information about the GUT type may be derived from the dominant decay mode. 
Supersymmetric GUTs \cite{Dimopoulos:1981dw,Bajc20161} prefer the presence of kaons in the products of the decay, e.g. $p\rightarrow K^+\bar{\nu}$  (see fig \ref{fig:MandatoryFeynmannDiagrams}, left).
%%%%%%%%%%%%%%%% Find theory papers!!!!!! %%%%%%%%%%
Gauge mediation GUTs, in which new gauge bosons are introduced that allow for the transformation of quarks into leptons, and vice versa, prefer the mode $p\rightarrow e^+\pi^0$ (see fig \ref{fig:MandatoryFeynmannDiagrams}, right).



\begin{figure}[hbpt]
\centering
\includegraphics[width=6.5in]{images/MandatoryFeynmannDiagrams.png}
\caption{Feynman diagrams for proton decay ``golden modes": $p \rightarrow K^+ \bar{\nu}$ for supersymmetric GUTs on the left and  $p \rightarrow e^+ \pi^0$ for gauge-mediation GUTs  on the right.}
\label{fig:MandatoryFeynmannDiagrams}
\end{figure}

The key elements for a rare decay experiment are: massive active volume, long exposure, high identification efficiency and low background. 
%The limit to proton lifetime in case of absence of signal and backgrounds is set by calculating
%$$\tau/B > M\times \epsilon\times T \times 10^{32},$$ 
%where M is the detector mass in kton, $\epsilon$ the signal detection efficiency after cuts to suppress backgrounds (dependent on the considered decay mode), T is the exposure in years, B the assumed branching fraction for the considered mode and  $10^{32}$ is a factor accounting for the number of nucleons in a kton of material \cite{Bueno2007}.
Figure \ref{fig:PDKExperimentalLImit} shows the current best experimental limits on nucleon decay lifetime over branching ratio (dots). Historically, the dominant technology used in these searches has been water Cherenkov detectors: all the best experimental limits on every decay mode are indeed set by Super-Kamiokande \cite{PhysRevD.90.072005,PhysRevLett.115.121803}.  It is particularly important to notice that the kaon energy for the proton decay mode $p \rightarrow K^+ \bar{\nu}$ is under Cherenkov threshold.  Super-Kamiokande set the limit on the lifetime for the $p \rightarrow K^+ \bar{\nu}$ mode by  relying exclusively on photons from nuclear de-excitation. For this reason, an attractive alternative approach to identifying nucleon decay is the use of a Liquid Argon Time Projection Chamber (LArTPC). 

LArTPCs can complement nucleon decay searches in modes where water Cherenkov detectors are less sensitive, especially $p\rightarrow K^+\bar{\nu}$.
%The stars in Figure \ref{fig:PDKExperimentalLImit} show the projected sensitivities for new generation of nucleon decay experiments: DUNE (LArTPC) and Hyper-K (water Cherenkov).
According to \cite{Acciarri:Dune}, DUNE will have an active volume large enough, have sufficient shielding from the surface, and will run for lengths of time sufficient to compete with Hyper-K, opening up the opportunity for the discovery of nucleon decay. 

\begin{figure}[hbpt]
\centering
\includegraphics[width=6.5in]{images/PDKExperimentalLImit.png}
\caption{Proton decay lifetime limits from passed and future experiments.}
\label{fig:PDKExperimentalLImit}
\end{figure}

Of course, LArIAT tiny active volume makes it impossible for the experiment to place competitive limits on nucleon decay.  
However,  LArIAT provides excellent data to characterize kaons in LAr for the ``LAr golden mode" $p \rightarrow K^+ \bar{\nu}$. The result of these studies will affect future proton decay searches in LArTPCs. Previous work has been done to assess the potential identification efficiency for different decay modes in a LArTPC \cite{Bueno2007}, but, as the time of this  writing, no study of kaon selection efficiency in LArTPCs has been performed on data. 
The K$^+$-Ar interaction cross section has never been measured before and can affect the possibility of detecting and measuring kaons when produced in a proton decay event. 
Kaon interactions with argon can distort the kaon energy spectrum as well as change the topology of single kaon events. In a LArTPC, non-interacting kaons appear as straight tracks with a high ionization depositions at the end (Bragg peak). The topology of interacting kaons can be quite different. In case of elastic scattering, a distinct kink will be present in the track. In case of inelastic scattering the Bragg peak will not be present and additional tracks (pions) will populate the event.
Performing the total K$^+$-Ar cross section measurement on data serves the double purpose of identifying the rate of ``unusual" topologies (kinks and additional tracks) and of developing tools for kaon tracking in LAr.


\subsection{Previous Measurements: Lighter and Heavier Nuclei}
In general, data on kaon cross sections are  extremely scarce. The measurement of kaon interaction cross section would bring the additional benefit of reducing the uncertainties associated  with hadron interaction models adopted in MC simulations for argon targets.

The data on nuclear effects for specific hadronic final states are extremely scarce, resulting in big uncertainties in modeling final state interactions \cite{Drakoulakos:2004gn}. Figure \ref{fig:Friedmann} shows a 1997 measurement on several elements as performed by  Friedmann et al.  \cite{Friedman:1997eq}. As a reference, this paper measures a $\sigma_{Tot}$ for Si of  366.5  $\pm$  4.8 mb and a $\sigma_{Tot}$ for Ca of 494.6  $\pm$ 7.7 mb at 488 MeV/c.  The cross section for argon is expected to lie in between these two measurements. 
Additional data on the kaon cross section are provided by Bugg et al. \cite{PhysRev.168.1466}. Bugg performs a measurement of the total 
K$^+$ and K$^-$ cross sections on protons and deuterons over the range of 0.6-2.65 GeV/c, as well as a measurement of the total K$^+$ and K$^-$  cross sections on carbon for a number of momenta; the results of this paper on carbon are reported in Figure \ref{fig:Bugg}.




\begin{figure}
\captionsetup{justification=raggedright}  
\begin{minipage}[b]{.5\textwidth}  
  \centering  
\includegraphics[width=3in]{images/Lariat/Friedmann.png}
\end{minipage}%  
\begin{minipage}[b]{0.5\textwidth}  
  \centering  
\includegraphics[width=3in]{images/Lariat/Bugg.png}
\end{minipage}
\par
\begin{minipage}[t]{.53\textwidth}
\caption{Ratios between experimental and calculated cross sections as from \cite{Friedman:1997eq}. Top: Total cross sections. Bottom: reaction cross sections.}
\label{fig:Friedmann}
\end{minipage}%
\begin{minipage}[t]{.5\textwidth}  
\caption{Total K$^+$  and K$^-$ cross sections on carbon as from \cite{PhysRev.168.1466}.}
\label{fig:Bugg}
\end{minipage}  
\end{figure}



%%%%%%%%%%%%%%%%%%%%%%%%%%%%%%%%%%%%%%%%%%%%%%%%%%%%%%%
%% PRETTY EVENT DISPLAY WITH TEXT, NOT SURE IF USEFUL
%\begin{figure}[h!]
%\centering
%\includegraphics[width=6.5in]{images/Lariat/KLariat.png}
%\caption{LArIAT Data $K^+$ candidate. $K^+$ enters TPC, undergoes a hadronic scatter, and then decays into $\pi^+$ and $\pi^0$. The the $\pi^0$ decays into 2 photons while the $\pi^+$ stops quickly in the TPC. Collection plane view.}
%\label{Fig:KLariat}
%\end{figure}

%Fig \ref{Fig:KLariat} shows a $K^+$ candidate event in the LArIAT TPC. Following the kaon candidate track from left to right, two important elements are visible by eye: a change in the K momentum due to hadronic scatter and a Bragg peak by the end of the track due to an augment of ionization as the kaon slows down in the TPC. The track "kink" is only visible thanks to the millimetric spacial resolution of the TPC, while the Bragg shows the calorimetric power of this technology. The kaon in this event decays hadronically into $\pi^+$ and $\pi^0$. The the $\pi^0$ decays into 2 photons while the $\pi^+$ stops quickly in the TPC. The ability to distinguish the topology of this decay from the most frequent one, i.e. $K^+\rightarrow\mu^+\nu$, remarks the versatility of the LArTPC technology.
%%%%%%%%%%%%%%%%%%%%%%%%%%%%%%%%%%%%%%%%%%%%%%%%%%%%%%%




 \subsection{Kaon Interaction Cross Section for thin target in Geant4}
Since the kaon cross section in argon has never been measured before, the Geant4 Monte Carlo tunes kaon transportation in argon by extrapolation from lighter and heavier nuclei. As shown in the previous section,  kaon data on carbon are available and  can be used as a metric to evaluate the Geant4 prediction performances.  Figure \ref{fig:TrueCarbon} shows the total hadronic cross section for carbon implemented in Geant4 10.01.p3 overlaid with the Bugg and Friedman data. Unfortunately, the current version of Geant4 does not reproduce the data for carbon closely. On one hand, this evidence makes us even more wary when using the Monte Carlo in simulating the kaon-argon interactions. On the other, it further highlights the importance of kaon measurements.

The K$^{+}$Ar cross section implemented in Geant4 can still be used as a tool to validate the measurement methodology. For the considered energy range, the Geant4 inelastic model adopted to is ``BertiniCascade", while the elastic model ``hElasticLHEP".  Figure \ref{fig:TrueArgon} shows the total hadronic cross section for argon implemented in Geant4 10.01.p3 (solid lines) overlaid with the true MC cross section as obtained with the sliced TPC method (markers); the total cross section is shown in green,  the elastic cross section in blue and the inelastic cross section in red. The sliced TPC method described in sec \ref{sec:KXSStrategy}. For the methodology validation we use the true energy deposition for a pool of about 140000 kaons fired inside the LArIAT TPC active volume and a uniform slice length of 4.7 mm.  The nice agreement with the Geant4 distribution and the cross section  obtained with the sliced TPC method gives us confidence in the  validity of the methodology. 
        
     
\begin{figure}
\captionsetup{justification=raggedright}  
\begin{minipage}[b]{.53\textwidth}  
  \centering  
\includegraphics[width=3in]{images/Lariat/CarbonG4.png}
\end{minipage}%  
\begin{minipage}[b]{0.53\textwidth}  
  \centering  
\includegraphics[width=3in]{images/Lariat/KaonTrueXS.png}
\end{minipage}
\par
\begin{minipage}[t]{.53\textwidth}
\caption{total hadronic cross section for carbon implemented in Geant4  10.01.p3  with overlaid with the Bugg and Frideman data.}
\label{fig:TrueCarbon}
\end{minipage}%
\begin{minipage}[t]{.5\textwidth}  
\caption{Hadronic cross sections for argon implemented in Geant4 10.01.p3 (solid lines) overlaid the true MC cross section as obtained with the sliced TPC method (markers). The total cross section is shown in green,  the elastic cross section in blue and the inelastic cross section in red.}
\label{fig:TrueArgon}
\end{minipage}  
\end{figure}






%%%%%%%%%%%%%%%%%%%%%%%%%%%%%%%%%
%% SECTION 2: Data Samples
%%%%%%%%%%%%%%%%%%%%%%%%%%%%%%%%%
\clearpage
\newpage
\section{Analysis strategy}\label{sec:kaonAnalysis} 
 
In this section, we will give an overview of the $K^{+}$-Ar total cross-section analysis strategy. 
The strategy can be summarize in 3 steps:
\begin{itemize}
\item[1.] Identification of kaon candidates in the beam line.
\item[2.] Application of the ``thin-slice'' method.
\item[3.] Identification and treatment of the slices containing a decaying kaon.
\end{itemize}

\subsection{Identification of Kaons in the beam line}

These candidates constitute the pool of events used for the cross section measurement.

\label{sec:BeamlineKStrategy}
\subsection{The thin-slice method}
\label{sec:KXSStrategy}
\subsubsection{Cross Section on Thin Target}\label{sec:thinTargetXS}
Interaction cross sections on thin target are a classic nuclear physics measurement with a well established methodology. A thin target is a target formed by a slab of material containing many uniformly distributed diffusion centers, where  one center is not sitting in front of another.
A pictorial representation of a thin slice experiment is shown schematically in Figure \ref{fig:thinslice}

\begin{figure}[htb]
\centering
\includegraphics[scale=0.25]{./images/ThinTarget.png}
\caption{Representation of the thin target approximation as a ``thin slice'' of argon experiments.}
\label{fig:thinslice}
\end{figure}

 The survival probability of a kaon traveling through a slab of argon of depth {\it z} and density {\it n} is given by:

\begin{equation}
P_{surv} = e^{-\sigma_{tot}n z}
\end{equation} 
where $\sigma_{tot}$ is the total cross section per nucleon (in $cm^2$), {\textit
{z}} is the target thickness (in cm) along the incident kaon direction, and {\textit
{n}} is the scattering center density in the target, $n=\frac{\rho N_{A} }{A}$ (in $cm^{-3}$). The interaction probability is then $P_{int} = 1 - P_{surv}$. $P_{int}$ is experimentally measured as the ratio of the number of interacting kaons $N_{int}$ over the number of incident kaons $N_{inc}$:
\begin{equation}
P_{int}=\frac{N_{int}}{N_{inc}}=1-e^{-\sigma_{tot}n z}.
\end{equation}

In practices, this assumption of thin target holds true if the target is several order of magnitude smaller than the interaction length. Mathematically speaking, this assumption implies $z\rightarrow\delta z$. Thus, it is possible to Taylor expand the exponential and  solve for the total cross section as a function of energy, $\sigma_{tot}(E)$:
\begin{equation}\label{calc_sigma1}
\frac{N_{int}}{N_{inc}}=1-e^{-\sigma_{tot}n z}\simeq 1-(1-\sigma_{tot}n\delta z + o(\delta z^2)) 
\end{equation}
\begin{equation}\label{calc_sigma2}
\sigma_{tot}(E) \simeq \frac{1}{n\delta z} \Big(\frac{N_{int}}{N_{inc}}\Big) \text{ 	when $z\rightarrow\delta z$}.
%N_{int}(z,E)=(1-N_{inc}e^{-\sigma_{tot}(E)nz})
\end{equation}

In order to measure the cross section, a thin target experiment would simply count the number of incident kaons and the number of surviving kaons.

\subsubsection{Not-so-thin target: sliced TPC}
\label{sec:thick}
The LArIAT TPC, with its 90-cm thick active volume, is not a thin target. Nevertheless, the combination of fine-grained tracking and precise calorimetric information allow us to treat the active volume as a sequence of 240 adjacent thin targets. This technique, called the ``sliced TPC" method, allows to measure the kaon cross section as a function of energy.  In LArIAT, the two wire planes are each made of 240 wires oriented at +/- $60^{\circ}$ with a wire pitch of 4 mm; these planes collect signals proportional to the energy loss of the kaon in a $60^{\circ}$-inclined 4~mm thin slab of liquid argon. Thus, one can think of the TPC as being divided into $\sim$240 slices along the direction of the incident particles ({\textit
{z}} axis) with a spacing $\Delta${\textit
{z}} = 4 mm/sin($60^{\circ}$) $\approx$ 4.5~mm, as shown in Fig.~\ref{fig:slicedtpc}. 
Each slice can be now considered an independent ``thin target" experiment and the cross section calculation in Eq. \ref{calc_sigma2} can be iteratively applied. The kinetic energy of the kaon entering the TPC is determined by measuring its momentum with the tertiary beamline and assuming the kaon mass as mass hypothesis. At each given slice, the kaon incident kinetic  is then determined by subtracting its calorimetric energy released in the previous slice from the total kinetic energy at that slice. Thus, it is possible to perform a differential cross section measurement as a function of the energy because the kaon kinetic energy $K.E._{slice}$  is known before entering each slice. \\ 
When the kaon enters a slice, it contributes to $N_{inc}$ for the energy bin corresponding to  its kinetic energy  that slice. If it interacts in the slice, it also contributes to $N_{int}$ in the appropriate energy bin. If it does not interact, the kaon proceeds to the next slice and the counting is repeated for its new kinetic energy.\\

The uncertainty for each energy bin is calculated by error propagation from the uncertainty on $N_{incident}$ and $N_{interacting}$. 
Since the number of incident pions in each slice is given by a simple counting, it is safe to assume that $N_{incident}$ is distributed as a poissonian with mean and $\sigma^2$ equal to $N_{incident}$ in each bin.  
On the other hand, $N_{interacting}$ follows a binomial distribution: the particle in a given energy bin might or might not interact.  The interaction probability $p$ is $\frac{ N_{interacting}}{N_{incident}}$ and the number of tries $n$ is $N_{incident}$. 
So, the square of the variance for the binomial is given by  $$\sigma^2 = np(1-p) =  N_{incident}\frac{ N_{interacting}}{N_{incident}} (1-\frac{ N_{interacting}}{N_{incident}}) = N_{interacting}(1-\frac{ N_{interacting}}{N_{incident}}).$$

$N_{incident}$ and $N_{interacting}$ are not independent.
The uncertainty on the cross section is thus calculated as 
\begin{equation}
\delta\sigma_{tot}(E) = \sigma_{tot}(E) \Big(\frac{\delta N_{interacting}}{N_{interacting}}+\frac{\delta N_{incident}}{N_{incident}}\Big) 
\end{equation}
where:
\begin{eqnarray}
\delta N_{incident} = \sqrt[]{N_{incident}} \\
\delta N_{interacting} = \sqrt[]{N_{interacting}(1-\frac{ N_{interacting}}{N_{incident}})}.
\end{eqnarray}

%\textcolor{blue}{Sketch of sliced tpc technique?}
\begin{figure}[htpb]
\centering
\includegraphics[scale=1.25]{images/Lariat/SlicedTPC.png}\\
\caption{Sketch of Sliced TPC approach.}
\label{fig:slicedtpc}
\end{figure}

It is worth noticing an important difference between the procedure utilized by LArIAT to measure the total hadronic kaon cross section and the procedure used by other experiments in neutrino cross section measurements. In the latter, one needs to correct for the detector inefficiency in identifying neutrinos. In our measurement,  we need do not need to efficiency correct for the beam line candidates which we are not able to identify in the TPC. This is because the cross section calculation in Eq. \ref{calc_sigma2}  relies on measuring the ratio $\frac{ N_{interacting}}{N_{incident}}$, where both these numbers are drawn from tracked kaons in the TPC.

The sliced TPC technique was tested by comparing the results of this method with the Geant 4.10.1.p3 prediction of the total hadronic interaction cross section ($K^{+}$, Ar)  with Bertini Cascade model.
Fig.~\ref{fig:TrueArgon} shows the resulting total ${K^+}$ cross section extracted by the sliced TPC technique; it agrees well with the Geant 4  cross section.  This comparison demonstrates the power of the sliced TPC method for the measurement of the ($K^{+}$, Ar) cross section in LArIAT TPC geometry. 

%\begin{figure}[h!]
%\centering
%\includegraphics[scale=0.45]{./images/compare_new.png}
%\caption{$K^+$ on Ar total cross section dependence on the kinetic energy from MC simulations: comparison between the Geant4 prediction for a LAr thin target and cross section measurement with the sliced TPC technique based on MC truth energy deposited.}
%\label{fig:xsplot}
%\end{figure}



\subsection{Slices containing decays}
We address here an important difference between the thin target and the thick target experiments. While the fraction of kaon decay in the thin target is negligible, both in flight and at rest decays play an important role in the thick target case. 
Kaon decay proceeds by the weak interaction; since our goal is to measure the hadronic cross section, slices containing decaying kaons must not contribute to the number of interacting kaons. If one only considers the endpoint of the primary kaon track without identification of the decay, this slice will  wrongly contribute to the $N_{interacting}$ counting. 
In case of kaon decay at rest, the kinetic energy is ideally zero, s shown in figure \textcolor{red}{ADD INTERACTION FIGURE!!!!}. A bragg peak towards the end of the kaon track is also formed as the kaon comes to a stop. A simple way to eliminate the contribution from slices containing kaon decay at rest is putting a lower bound to the energy range of the cross section. We decide to measure the cross section in the region between 100 MeV to 2000 MeV.

Kaon decay in flight are more tricky, since they can happen at any energy. A distinction between interaction and decay is thus needed. 





 Kaon decay tagging is discussed the next section. 





\clearpage
\newpage 
\section{Data / MC Samples}\label{sec:DataSamples}

This section outlines the data and Monte Carlo sets used in this analysis. \\
For the data, we are using a set which spans all of Run-II positive polarity data. Details on this sample can be found in Section \ref{sec:data}. For the simulation, we use the G4Beamline Monte Carlo (Section \ref{sec:G4Beamline}) and the Data Driven single particle Monte Carlo (DDMC, Section \ref{sec:DDMCSamples}). 


%%%%%%%%%%%%%%%%%%%%%%%%%%%%%%%%%%%%%%%%%%%%%%
\subsection{Data}\label{sec:data}
%%%%%%%%%%%%%%%%%%%%%%%%%%%%%%%%%%%%%%%%%%%%%%



LArIAT successfully ran for 9 weeks in 2015 (Run I) and 24 weeks in 2016 (Run II). Some spectacular Kaon interactions were found in data from Run I (see Figure \ref{fig:MCdata} and its great agreement with the MC),
but the Kaon statistics in Run I is not enough to perform a cross section analysis. Figure \ref{fig:BeamComp} shows the reason behind the low statistics: this figure represents the tertiary beam composition of one of the first data runs. Two aspects of this plot are particularly notable. Firstly, Kaons are very few in the beam. However, the few Kaons produced are in the correct range of momentum for proton decay studies (compare the momentum on Figure \ref{fig:KGenie}).  LArIAT Run II provides enough statistic to measure Kaon cross section. 
\begin{figure}[hpbt]
\centering
\includegraphics[width=6in]{images/Lariat/KDataMC}
\caption{Direct comparson between a Kaon event in LArIAT Run I data and in LArIAT MC. }
\label{fig:MCdata}
\end{figure}


\begin{figure}[h!]
\centering
\begin{minipage}{0.45\textwidth}
\centering
\includegraphics[width=3.5in]{images/Lariat/Beam}
\caption{Particle spectrum at the TPC produced with the LArIAT 8 GeV tertiary beam.}
\label{fig:BeamComp}
\end{minipage}\hfill
\centering
\begin{minipage}{0.45\textwidth}
\centering
\includegraphics[width=3in]{images/Lariat/KaonGenie}
\caption{Momentum distribution for the Kaon in the $p \rightarrow K^{+} \bar\nu$ mode proton decay as simulated by GENIE.}
\label{fig:KGenie}
\end{minipage}
\end{figure}




The Run-II data use the definitions \href{https://redmine.fnal.gov/redmine/projects/lardbt/wiki/Recommended_SAM_Datasets}{oulined on this Wiki page} and summarized in Table \ref{tab:datasamples}.

\begin{center}
\begin{table}[htb]
	\begin{center}
	%\resizebox{0.95\textwidth}{!}{%
	\begin{tabular}{|c|c|c|}
	\multicolumn{3}{c}{\textbf{Summary of Data Samples}} \\
	\hline \hline
	 Run Period & Data Set Definition & Samweb Meaning \\
%	\hline
%	 &  & \verb!defname: TPC_voltages_nominal! \\
%	\hline
%	 &  & \verb!TPC_MaxGainAndFilter! \\
%	\hline
%	Run-I & \verb!Lovely1_Neg_RunI_elenag_v02! & \verb!TPC_nominal_read_out_and_timing!  \\
%	\hline
%	 & & \verb!BothTOF_OnAndReadOut!  \\
%	\hline
%	 & & \verb!AllMWPC_OnAndReadOut!  \\
%	 \hline
%	 & & \verb!lariat_mid_f_mc7anb < 0! \\
%	\hline
%	\line
%	 & & \verb!run_number >= 8000 and run_number <= 10226! \\
     \hline	
	&  & \verb!defname: TPC_voltages_nominal! \\
	\hline
	 &  & \verb!TPC_MaxGainAndFilter! \\
	\hline
	Run-II & \verb!Lovely1_Pos_RunII_elenag_v04! & \verb!TPC_nominal_read_out_and_timing!  \\
	\hline
	 & & \verb!BothTOF_OnAndReadOut!  \\
	\hline
	 & & \verb!AllMWPC_OnAndReadOut!  \\
	 \hline
	 & & \verb!lariat_mid_f_mc7anb > 0! \\
	 	 \hline
	 & & \verb!create_date < '2017-06-02'! \\

	 \hline
	\end{tabular}%}
	\caption{Summary of the data samples used for this analysis. }
	\label{tab:datasamples}
	\end{center}
\end{table}
\end{center}

The relevant Samweb definitions listed in Table \ref{tab:datasamples} which require some explaining are defined as:

\begin{itemize}
\item \textbf{TPC Voltages Nominal}: Requires the cathode to be at greater than 23 kV, the collection plane wires voltage to be between 320 and 350 V, the induction plane voltage to be between -10 and -20 V, and the shield plane voltage to be greater than -310 V

\item \textbf{TPC MaxGainAndFilter}: Requires the ASIC configuration to be set as ``3'' for both the filter and the gain setting

\item \textbf{TPC Nominal Read Out and Timing}: Requires the readout of the TPC was enabled, the recorded number of time ticks is 3072, and the delay of 36900 was set on the v1495 (trigger card).
\item \textbf{lariat\_mid\_f\_mc7anb \textgreater 0} : Requires the polarity of the magnets to be positive
\item \textbf{create\_date \textless  2017-06-02}: Avoids the introduction of run 3 data and newly sliced data
\end{itemize}


It is important to provide a break down of the beam conditions for the period of data taking because the beam composition, hence the kaon content, varies according to the energy of the secondary beam and the strength of the magnetic field inside the magnets. Table \ref{tab:beamConditions} shows a break down of the beam conditions for both the beam data selected by  \verb!Lovely1_Pos_RunII_elenag_v04! and the events that pass the kaon candidates selection.


\begin{table}[]
\centering
\caption{Break down of beam conditions for Run-II positive polarity. $I$ is the value of the current in the magnets and $E$ is the energy of the secondary beam.  }
\label{tab:beamConditions}
\begin{tabular}{l|l|l|l|l|}
\cline{2-5}
                                       & \multicolumn{2}{l|}{Lovely1\_Pos\_RunII\_elenag\_v04} & \multicolumn{2}{l|}{Kaon candidate sample} \\ \cline{2-5} 
                                       & Run \% & Event \% & Run \% & Event \% \\ \hline
\multicolumn{1}{|l|}{I = + 100 A, E = 64 GeV}   &   61.8          &        75.0   & 75.1  &  90.35   \\ \hline
\multicolumn{1}{|l|}{I =   + 60 A, E = 64 GeV}   &    30.1         &         23.8  & 24.9  &   9.65     \\ \hline
\multicolumn{1}{|l|}{Info not available}              &      8.1         &           1.2   &    0.0 &       0.0               \\ \hline
\end{tabular}
\end{table}



%%%%%%%%%%%%%%%%%%%%%%%%%%%%%%%%%%%%%%%%%%%%%%%%%%%%%%%%%%%%
\subsection{Monte Carlo Samples}\label{sec:MCSamples}
For the simulation of the tertiary beam, we use a combination of two MC generators: the G4Beamline Monte Carlo and the Data Driven single particle Monte Carlo (DDMC).   We use the G4Beamline MC to calculate the particle composition of the beam just before the cryostat. In order to simulate the beam line particles after Wire-Chamber 4, we use the DDMC. 

\subsubsection{G4Beamline }\label{sec:G4Beamline}
At the moment of this writing,  G4Beamline simulates transportation of particles through the beam line from the LArIAT target until ``Big Disk'', a fictional, void detector located just before the cryostat. The responses of  the beam line detector are not simulated. 

The two beam conditions relevant for this analysis are simulated: secondary beam energy of 64 GeV, positive polarity magnet with current of 100 A and 60 A. Figure \ref{fig:beamspectrum} shows the tertiary beam spectra for the 64 GeV and 100 A condition on the left and for the 64 GeV and 60 A condition on the right.
In Table \ref{tab:beamcomp2}, the beam composition is given in terms of percentage of different particle species per spill for positive polarity. The values reported are the weighted average on the two beam conditions considered. The weights are calculated according to the fourth column of Table \ref{tab:beamConditions}. 

\begin{table}[ht!]
\centering
\begin{tabular}{|l|l|l|l|l|l|l|}
\hline
                   & $\pi^+$ & $e^+$ & $\gamma$ & $\mu^+$ & $K^+$ & p \\ \hline
Beam Composition (\%) &    42.8     &  30.1     &    8.6      &    2.1     &    0.057    &    16.2            \\ \hline
\end{tabular}
\caption{Beam Composition - Positive polarity configuration (from MC)}
\label{tab:beamcomp2}
\end{table}



\begin{figure}[htb]
\begin{center}
%\includegraphics[scale=0.45]{}
\end{center}
\caption{G4Beamline tertiary beam  predicted spectra for positive 100 Amp, 64 GeV target energy with data overlaid (left). G4Beamline tertiary beam  predicted spectra for positive 60 Amp, 64 GeV target energy with data overlaid (right).}
\label{fig:beamspectrum}
\end{figure}



\subsubsection{Data Driven Single Particle MC (DDMC) }\label{sec:DDMCSamples}
%%%%%%%%%%%%%%%%%%%%%%%%%%%%%%%%%%%%%%%%%%%%%%%%%%%%%%%%%%%%
The DDMC uses the data momentum and position at wire chamber 4 to derive its initial conditions. The details of these samples and where they can be found are given in \href{https://docs.google.com/spreadsheets/d/1_0kNCKBIIx53f6vopqN2OijtcTICHD9rDvN_YKGH2mI/edit?usp=sharing}{this data production spreadsheet}.
The precise details of how the Monte Carlo used in this study are given in \href{https://lartpc-docdb.fnal.gov:441/cgi-bin/ShowDocument?docid=2054}{docDB-2054} and  \href{https://lartpc-docdb.fnal.gov:441/cgi-bin/ShowDocument?docid=2056}{dobDB-2056}, a summary of which is presented here. 

The Data Driven Monte Carlo (DDMC) uses data quantities for a sample of Wire-Chamber tracks to derive the momentum ($P_x, P_y, P_z$) and position at WC4 $X, Y$ distributions that are seen during a particular running period and/or running condition. Using those data derived distributions, it then launches single particle MC from $z = -100$~cm (the location of the fourth wire chamber) with these distributions as a template. An illustration of this procedure is shown in Figure \ref{fig:DDMC} with the results of the DDMC generation compared to a sample of wire chamber track data. Using this technique ensures the MC and data have very similar momentum, position and angular distributions at Wire-Chamber 4 and allow us to calibrate the energy loss upstream of the TPC as precisely as possible. The DDMC is a single particle Monte Carlo: the beam pile-up is not simulated.

\begin{figure}[htb]
\centering
\includegraphics[width=0.70\textwidth]{images/DDMC.png}
\caption{Illustration of the technique where the wire chamber track initial angular and momentum distributions are used to generate the single particle MC.}
\label{fig:DDMC}
\end{figure}

Table \ref{tab:MCSampleGen} lists the various MC samples that were generated for this analysis. 

\begin{table}[htb]
	\begin{center}
	\resizebox{0.95\textwidth}{!}{%
	\begin{tabular}{|c|c|c|}
	\hline
	  \textbf{DDMC Sample} & Original Data Distribution & Number of Events Generated  \\
	  	\hline
	Run-II $\pi^{+}$ & $\pi, \mu, e$ Mass Filter / Picky WC-Track & \\
	Run-II $\mu^{+}$ & $\pi, \mu, e$ Mass Filter / Picky WC-Track &  \\
	Run-II $e^{+}$ & $\pi, \mu, e$ Mass Filter / Picky WC-Track & \\
	Run-II $K^{+}$ & $K^{+}$ Mass Filter / Picky WC-Track & \\
	Run-II $p$ & $p$ Mass Filter / Picky WC-Track & \\
	\hline
	\end{tabular}}
	\caption{Summary of MC generated for the analysis.} \label{tab:MCSampleGen}
	\end{center}
\end{table}

\textcolor{red}{CHECK WITH THE BEAMLINE MC IF WE REALLY NEED THIS}
In addition to this sample of DDMC, a sample of photons is also generated since as is shown in Table \ref{tab:beamcomp1} a small but non-negligible portion of the beam will have photons entering the TPC. This sample is generated with a flat momentum spectrum between 0 MeV and 2000 MeV with a Gaussian angular distribution of $\pm$5 degrees about the beam direction. The photon momentum spectrum is then re-weighted by the momentum spectrum of the corresponding run period it is being simulated for. This approximation allows us to estimate the contamination due to photons from MC with a reasonable assumption of their spectrum.



\newpage
%
%%%%%%%%%%%%%%%%%%%%%%%%%%%%%%%%%%%%%%%%%%%%%%%%%%%%%%%%%
\section{Kaon candidate selection in the beamline}\label{sec:BeamlineSelection}
\subsection{Overview of the LArIAT Beamline}\label{sec:BeamlineOverview}
%%%%%%%%%%%%%%%%%%%%%%%%%%%%%%%%%%%%%%%%%%%%%%%%%%%%%%%%%

%%%%%%%%%%%%%%%%%%%%%%%%%%%%%%%%%%%%%%%%%%%%%%%%%%%%%%%%%%%%%%%%%%%%%%%%%
\subsection{Beam composition}\label{sec:G4BeamlineMC}
%%%%%%%%%%%%%%%%%%%%%%%%%%%%%%%%%%%%%%%%%%%%%%%%%%%%%%%%%%%%%%%%%%%%%%%%%



\subsection{Data Selection Cut}\label{sec:DataSelectionCut}
\subsubsection{Beamline Candidates}\label{sec:BeamlineCandidates}

%%%%%%%%%%%%%%%%%%%%%%%%%%%%%%%%%%%%%%%%%%%%%%%%%%%%%%%%%%%%
For each sample listed in Table \ref{tab:datasamples}, we outline the event selection used to select this data.

\begin{itemize}
\item \textbf{Time Stamp Filter}

A filter is used to select events which occurred in time with the beam. These events typically coincide with the first 6 seconds of the beam spill, and therefore the events are filtered using the following LArIATsoft settings

\begin{verbatim}
tfilt:      @local::lariat_timestampfilter

# ====================================================================
# Specify range of events to select.  For Run I/II:
#   - pedestal events:  ~ 0.  - 1.2 sec
#   - beam events:      ~ 1.2 - 5.5 sec
#   - cosmic events:    ~ > 5.5 sec
#   (default selects ALL events)
physics.filters.tfilt.T1:                       1.2
physics.filters.tfilt.T2:                       5.5
physics.filters.tfilt.RequireRawDigits:         true

\end{verbatim}



\item \textbf{Beamline Reconstruction}

The standard LArIAT beamline reconstruction is used to select events which have a wire chamber track and TOF information in an individual event using the following modules.
\begin{verbatim}
### beamline elements ###

wctrack:     @local::lariat_wctrackbuilder
tof:         @local::lariat_tof
agcounter:   @local::lariat_aerogel
\end{verbatim}


\textbf{For Run-I we use these default parameters:}
\begin{verbatim} 
physics.producers.wctrack.PickyTracks:                          false
physics.producers.tof.HitThreshold:                           -10.0  
physics.producers.tof.HitDiffMeanUS:                            0.6  
physics.producers.tof.HitDiffMeanDS:                            1.0  
physics.producers.tof.HitMatchThresholdUS:                      3.0  
physics.producers.tof.HitMatchThresholdDS:                      6.0  
physics.producers.tof.HitWait:                                  20.
\end{verbatim}

\textbf{For Run-II we use these default parameters:}
\begin{verbatim} 
physics.producers.wctrack.PickyTracks:                          false
physics.producers.tof.HitThreshold:                             -3.
physics.producers.tof.HitDiffMeanUS:                            0.5  
physics.producers.tof.HitDiffMeanDS:                            0.4  
physics.producers.tof.HitMatchThresholdUS:                      3.0  
physics.producers.tof.HitMatchThresholdDS:                      6.0  
physics.producers.tof.HitWait:                                  20.
\end{verbatim}

We require the tracks reconstructed in the wire chamber satisfy the criteria known as a ``picky track''. ``Picky tracks'' correspond to tracks reconstructed using hits in all four wire chambers. In these events, one and only one hit in each wire chamber track can be reconstructed per event and the track satisfies a straightness requirement in the Y-Z plane. These tracks have more accurate measure of the particle momentum than the ``high yield'' (HY) tracks.  HY tracks only require hits in three out of four of the wire chamber tracks and can have multiple wire chamber hits reconstructed per event. HY tracks yield better statistics: a subset of HY tracks can be used as a orthogonal sample for the study of systematics. Details about wire chamber track reconstruction can be found in \cite{WCTrackReco}

\item \textbf{Particle Mass Filtering}
Using the beamline reconstruction, it is possible to calculate the mass of a given track, as shown in Figure \ref{fig:mass}. The classification of events into the different samples follows:

\begin{itemize}
\item \underline{$\pi, \mu, e$:} 0~MeV $<$ mass $<$ 350~MeV

\item \underline{kaon:} 350~MeV $<$ mass $<$ 650~MeV

\item \underline{proton:} 650~MeV $<$ mass $<$ 3000~MeV

\end{itemize}

\begin{figure}[htb]
\centering
\includegraphics[width=0.70\textwidth]{images/mass.png}
\caption{The mass plotted for a sample of Run-II events reconstructed in the beamline. The classification of the events into $\pi, \mu, e$, kaon, or proton is based on this distribution.}
\label{fig:mass}
\end{figure}

For this analysis we require 350~MeV $<$ mass $<$ 650~MeV to select a sample of $K$ candidates for further event selection. The full event reduction table for these cuts is presented in Section \ref{sec:Results}.

\end{itemize}


\subsubsection{Candidates}\label{sec:Candidates}
\subsubsection{Contamination from different particle species in the beam line }\label{sec:Contamination}
\section{TPC Candidates} \label{sec:TPC} 
\section{Systematics} \label{sec:Systematics} 
\section{Results}\label{sec:Results}

\newpage
%\section{Analysis method}
\subsection{Measurement Strategy -- TPC "slicing"}
\label{sec:KXSStrategy}
A thin target is a target formed by a slab of material containing many uniformly distributed diffusion centers, where  one center is not sitting in front of another. In this approximation, the survival probability of a kaon traveling through a slab of argon of depth {\it z} and density {\it n} is given by:

\begin{equation}
P_{surv} = e^{-\sigma_{tot}n z}
\end{equation} 
where $\sigma_{tot}$ is the total cross section per nucleon (in $cm^2$), {\emph{z}} is the target thickness (in cm) along the incident kaon direction, and {\emph{n}} is the scattering center density in the target, $n=\frac{\rho N_{A} }{A}$ (in $cm^{-3}$). Thus, the interaction probability is $P_{int} = 1 - P_{surv}$. $P_{int}$ is experimentally measured as the ratio of the number of interacting kaons $N_{int}$ over the number of incident kaons $N_{inc}$:
\begin{equation}
P_{int}=\frac{N_{int}}{N_{inc}}=1-e^{-\sigma_{tot}n z}.
\end{equation}

The assumption of thin target implies $z\rightarrow\delta z$. Thus, it is possible to Taylor expand the exponential and  solve for the total cross section as a function of energy, $\sigma_{tot}(E)$:
\begin{equation}\label{calc_sigma1}
\frac{N_{int}}{N_{inc}}=1-e^{-\sigma_{tot}n z}\simeq 1-(1-\sigma_{tot}n\delta z + o(\delta z^2)) 
\end{equation}
\begin{equation}\label{calc_sigma}
\sigma_{tot}(E) \simeq \frac{1}{n\delta z} \Big(\frac{N_{int}}{N_{inc}}\Big) \text{ 	when $z\rightarrow\delta z$}.
%N_{int}(z,E)=(1-N_{inc}e^{-\sigma_{tot}(E)nz})
\end{equation}


LArIAT, with its 90-cm thick active volume, is not a thin target. Nevertheless, the combination of fine-grained tracking and precise calorimetric information allow us to treat the active volume as a sequence of 240 adjacent thin targets. This technique, called the "sliced TPC" method, allows to measure the kaon cross section as a function of energy.  In LArIAT, the two wire planes are each made of 240 wires oriented at +/- $60^{\circ}$ with a wire pitch of 4 mm; these planes collect signals proportional to the energy loss of the kaon in a $60^{\circ}$-inclined 4~mm thin slab of liquid argon. One can thus think of the TPC as being divided into $\sim$240 slices along the direction of the incident particles ({\emph{z}} axis) with a spacing $\Delta${\emph{z}} = 4 mm/sin($60^{\circ}$) $\approx$ 4.5~mm, as shown in Fig.~\ref{fig:slicedtpc}. Each slice can be now considered a "thin target" and the cross section calculation in Eq.~\ref{calc_sigma} can be iteratively applied. It is possible to perform a differential cross section measurement as a function of the energy because the kaon kinetic energy $K.E._{slice}$  at the beginning of slice is known per each slice. The kaon kinetic energy entering the TPC is determined by measuring the particle momentum with the tertiary beamline and assuming the kaon mass as mass hypothesis. At each slice, the incident energy of the kaon is determined by subtraction of the calorimetric energy released by the particle in the previous slice.\\ 
When the particle enters a slice, it contributes to $N_{inc}$ in the energy bin corresponding to its kinetic energy in that slice. If it interacts in the slice, it also contributes to $N_{int}$ in the appropriate energy bin.\\

An important difference between the thin target approximation and the thick target case needs to be address. While in the first case, only the K-nucleus hadronic interactions are taken into account, other processes such as kaon decay, both in flight and at rest play an important role in the "thick" target case. Kaon decays are a physical background for this measurement. Kaon decay proceeds by the weak interaction; if one only considers the endpoint of the primary kaon track without identification of the decay, this process can be misidentified as a single strong interaction. Kaon decay tagging is discussed the next section. 


%\textcolor{blue}{Sketch of sliced tpc technique?}
\begin{figure}[htpb]
\centering
\includegraphics[scale=1.25]{images/Lariat/SlicedTPC.png}\\
\caption{Sketch of Sliced TPC approach.}
\label{fig:slicedtpc}
\end{figure}



%%%%%%%%%%%%%%%%%%%%%%%%%%%%%%%%%
%% SECTION 3:Data Samples
%%%%%%%%%%%%%%%%%%%%%%%%%%%%%%%%%
%\section{Data / MC Samples}\label{sec:DataSamples}

This section outlines the data and Monte Carlo sets used in this analysis. \\
For the data, we are using a set which spans all of Run-II positive polarity data. Details on this sample can be found in Section \ref{sec:data}. For the simulation, we use the G4Beamline Monte Carlo (Section \ref{sec:G4Beamline}) and the Data Driven single particle Monte Carlo (DDMC, Section \ref{sec:DDMCSamples}). 


%%%%%%%%%%%%%%%%%%%%%%%%%%%%%%%%%%%%%%%%%%%%%%
\subsection{Data}\label{sec:data}
%%%%%%%%%%%%%%%%%%%%%%%%%%%%%%%%%%%%%%%%%%%%%%



LArIAT successfully ran for 9 weeks in 2015 (Run I) and 24 weeks in 2016 (Run II). Some spectacular Kaon interactions were found in data from Run I (see Figure \ref{fig:MCdata} and its great agreement with the MC),
but the Kaon statistics in Run I is not enough to perform a cross section analysis. Figure \ref{fig:BeamComp} shows the reason behind the low statistics: this figure represents the tertiary beam composition of one of the first data runs. Two aspects of this plot are particularly notable. Firstly, Kaons are very few in the beam. However, the few Kaons produced are in the correct range of momentum for proton decay studies (compare the momentum on Figure \ref{fig:KGenie}).  LArIAT Run II provides enough statistic to measure Kaon cross section. 
\begin{figure}[hpbt]
\centering
\includegraphics[width=6in]{images/Lariat/KDataMC}
\caption{Direct comparson between a Kaon event in LArIAT Run I data and in LArIAT MC. }
\label{fig:MCdata}
\end{figure}


\begin{figure}[h!]
\centering
\begin{minipage}{0.45\textwidth}
\centering
\includegraphics[width=3.5in]{images/Lariat/Beam}
\caption{Particle spectrum at the TPC produced with the LArIAT 8 GeV tertiary beam.}
\label{fig:BeamComp}
\end{minipage}\hfill
\centering
\begin{minipage}{0.45\textwidth}
\centering
\includegraphics[width=3in]{images/Lariat/KaonGenie}
\caption{Momentum distribution for the Kaon in the $p \rightarrow K^{+} \bar\nu$ mode proton decay as simulated by GENIE.}
\label{fig:KGenie}
\end{minipage}
\end{figure}




The Run-II data use the definitions \href{https://redmine.fnal.gov/redmine/projects/lardbt/wiki/Recommended_SAM_Datasets}{oulined on this Wiki page} and summarized in Table \ref{tab:datasamples}.

\begin{center}
\begin{table}[htb]
	\begin{center}
	%\resizebox{0.95\textwidth}{!}{%
	\begin{tabular}{|c|c|c|}
	\multicolumn{3}{c}{\textbf{Summary of Data Samples}} \\
	\hline \hline
	 Run Period & Data Set Definition & Samweb Meaning \\
%	\hline
%	 &  & \verb!defname: TPC_voltages_nominal! \\
%	\hline
%	 &  & \verb!TPC_MaxGainAndFilter! \\
%	\hline
%	Run-I & \verb!Lovely1_Neg_RunI_elenag_v02! & \verb!TPC_nominal_read_out_and_timing!  \\
%	\hline
%	 & & \verb!BothTOF_OnAndReadOut!  \\
%	\hline
%	 & & \verb!AllMWPC_OnAndReadOut!  \\
%	 \hline
%	 & & \verb!lariat_mid_f_mc7anb < 0! \\
%	\hline
%	\line
%	 & & \verb!run_number >= 8000 and run_number <= 10226! \\
     \hline	
	&  & \verb!defname: TPC_voltages_nominal! \\
	\hline
	 &  & \verb!TPC_MaxGainAndFilter! \\
	\hline
	Run-II & \verb!Lovely1_Pos_RunII_elenag_v04! & \verb!TPC_nominal_read_out_and_timing!  \\
	\hline
	 & & \verb!BothTOF_OnAndReadOut!  \\
	\hline
	 & & \verb!AllMWPC_OnAndReadOut!  \\
	 \hline
	 & & \verb!lariat_mid_f_mc7anb > 0! \\
	 	 \hline
	 & & \verb!create_date < '2017-06-02'! \\

	 \hline
	\end{tabular}%}
	\caption{Summary of the data samples used for this analysis. }
	\label{tab:datasamples}
	\end{center}
\end{table}
\end{center}

The relevant Samweb definitions listed in Table \ref{tab:datasamples} which require some explaining are defined as:

\begin{itemize}
\item \textbf{TPC Voltages Nominal}: Requires the cathode to be at greater than 23 kV, the collection plane wires voltage to be between 320 and 350 V, the induction plane voltage to be between -10 and -20 V, and the shield plane voltage to be greater than -310 V

\item \textbf{TPC MaxGainAndFilter}: Requires the ASIC configuration to be set as ``3'' for both the filter and the gain setting

\item \textbf{TPC Nominal Read Out and Timing}: Requires the readout of the TPC was enabled, the recorded number of time ticks is 3072, and the delay of 36900 was set on the v1495 (trigger card).
\item \textbf{lariat\_mid\_f\_mc7anb \textgreater 0} : Requires the polarity of the magnets to be positive
\item \textbf{create\_date \textless  2017-06-02}: Avoids the introduction of run 3 data and newly sliced data
\end{itemize}


It is important to provide a break down of the beam conditions for the period of data taking because the beam composition, hence the kaon content, varies according to the energy of the secondary beam and the strength of the magnetic field inside the magnets. Table \ref{tab:beamConditions} shows a break down of the beam conditions for both the beam data selected by  \verb!Lovely1_Pos_RunII_elenag_v04! and the events that pass the kaon candidates selection.


\begin{table}[]
\centering
\caption{Break down of beam conditions for Run-II positive polarity. $I$ is the value of the current in the magnets and $E$ is the energy of the secondary beam.  }
\label{tab:beamConditions}
\begin{tabular}{l|l|l|l|l|}
\cline{2-5}
                                       & \multicolumn{2}{l|}{Lovely1\_Pos\_RunII\_elenag\_v04} & \multicolumn{2}{l|}{Kaon candidate sample} \\ \cline{2-5} 
                                       & Run \% & Event \% & Run \% & Event \% \\ \hline
\multicolumn{1}{|l|}{I = + 100 A, E = 64 GeV}   &   61.8          &        75.0   & 75.1  &  90.35   \\ \hline
\multicolumn{1}{|l|}{I =   + 60 A, E = 64 GeV}   &    30.1         &         23.8  & 24.9  &   9.65     \\ \hline
\multicolumn{1}{|l|}{Info not available}              &      8.1         &           1.2   &    0.0 &       0.0               \\ \hline
\end{tabular}
\end{table}



%%%%%%%%%%%%%%%%%%%%%%%%%%%%%%%%%%%%%%%%%%%%%%%%%%%%%%%%%%%%
\subsection{Monte Carlo Samples}\label{sec:MCSamples}
For the simulation of the tertiary beam, we use a combination of two MC generators: the G4Beamline Monte Carlo and the Data Driven single particle Monte Carlo (DDMC).   We use the G4Beamline MC to calculate the particle composition of the beam just before the cryostat. In order to simulate the beam line particles after Wire-Chamber 4, we use the DDMC. 

\subsubsection{G4Beamline }\label{sec:G4Beamline}
At the moment of this writing,  G4Beamline simulates transportation of particles through the beam line from the LArIAT target until ``Big Disk'', a fictional, void detector located just before the cryostat. The responses of  the beam line detector are not simulated. 

The two beam conditions relevant for this analysis are simulated: secondary beam energy of 64 GeV, positive polarity magnet with current of 100 A and 60 A. Figure \ref{fig:beamspectrum} shows the tertiary beam spectra for the 64 GeV and 100 A condition on the left and for the 64 GeV and 60 A condition on the right.
In Table \ref{tab:beamcomp2}, the beam composition is given in terms of percentage of different particle species per spill for positive polarity. The values reported are the weighted average on the two beam conditions considered. The weights are calculated according to the fourth column of Table \ref{tab:beamConditions}. 

\begin{table}[ht!]
\centering
\begin{tabular}{|l|l|l|l|l|l|l|}
\hline
                   & $\pi^+$ & $e^+$ & $\gamma$ & $\mu^+$ & $K^+$ & p \\ \hline
Beam Composition (\%) &    42.8     &  30.1     &    8.6      &    2.1     &    0.057    &    16.2            \\ \hline
\end{tabular}
\caption{Beam Composition - Positive polarity configuration (from MC)}
\label{tab:beamcomp2}
\end{table}



\begin{figure}[htb]
\begin{center}
%\includegraphics[scale=0.45]{}
\end{center}
\caption{G4Beamline tertiary beam  predicted spectra for positive 100 Amp, 64 GeV target energy with data overlaid (left). G4Beamline tertiary beam  predicted spectra for positive 60 Amp, 64 GeV target energy with data overlaid (right).}
\label{fig:beamspectrum}
\end{figure}



\subsubsection{Data Driven Single Particle MC (DDMC) }\label{sec:DDMCSamples}
%%%%%%%%%%%%%%%%%%%%%%%%%%%%%%%%%%%%%%%%%%%%%%%%%%%%%%%%%%%%
The DDMC uses the data momentum and position at wire chamber 4 to derive its initial conditions. The details of these samples and where they can be found are given in \href{https://docs.google.com/spreadsheets/d/1_0kNCKBIIx53f6vopqN2OijtcTICHD9rDvN_YKGH2mI/edit?usp=sharing}{this data production spreadsheet}.
The precise details of how the Monte Carlo used in this study are given in \href{https://lartpc-docdb.fnal.gov:441/cgi-bin/ShowDocument?docid=2054}{docDB-2054} and  \href{https://lartpc-docdb.fnal.gov:441/cgi-bin/ShowDocument?docid=2056}{dobDB-2056}, a summary of which is presented here. 

The Data Driven Monte Carlo (DDMC) uses data quantities for a sample of Wire-Chamber tracks to derive the momentum ($P_x, P_y, P_z$) and position at WC4 $X, Y$ distributions that are seen during a particular running period and/or running condition. Using those data derived distributions, it then launches single particle MC from $z = -100$~cm (the location of the fourth wire chamber) with these distributions as a template. An illustration of this procedure is shown in Figure \ref{fig:DDMC} with the results of the DDMC generation compared to a sample of wire chamber track data. Using this technique ensures the MC and data have very similar momentum, position and angular distributions at Wire-Chamber 4 and allow us to calibrate the energy loss upstream of the TPC as precisely as possible. The DDMC is a single particle Monte Carlo: the beam pile-up is not simulated.

\begin{figure}[htb]
\centering
\includegraphics[width=0.70\textwidth]{images/DDMC.png}
\caption{Illustration of the technique where the wire chamber track initial angular and momentum distributions are used to generate the single particle MC.}
\label{fig:DDMC}
\end{figure}

Table \ref{tab:MCSampleGen} lists the various MC samples that were generated for this analysis. 

\begin{table}[htb]
	\begin{center}
	\resizebox{0.95\textwidth}{!}{%
	\begin{tabular}{|c|c|c|}
	\hline
	  \textbf{DDMC Sample} & Original Data Distribution & Number of Events Generated  \\
	  	\hline
	Run-II $\pi^{+}$ & $\pi, \mu, e$ Mass Filter / Picky WC-Track & \\
	Run-II $\mu^{+}$ & $\pi, \mu, e$ Mass Filter / Picky WC-Track &  \\
	Run-II $e^{+}$ & $\pi, \mu, e$ Mass Filter / Picky WC-Track & \\
	Run-II $K^{+}$ & $K^{+}$ Mass Filter / Picky WC-Track & \\
	Run-II $p$ & $p$ Mass Filter / Picky WC-Track & \\
	\hline
	\end{tabular}}
	\caption{Summary of MC generated for the analysis.} \label{tab:MCSampleGen}
	\end{center}
\end{table}

\textcolor{red}{CHECK WITH THE BEAMLINE MC IF WE REALLY NEED THIS}
In addition to this sample of DDMC, a sample of photons is also generated since as is shown in Table \ref{tab:beamcomp1} a small but non-negligible portion of the beam will have photons entering the TPC. This sample is generated with a flat momentum spectrum between 0 MeV and 2000 MeV with a Gaussian angular distribution of $\pm$5 degrees about the beam direction. The photon momentum spectrum is then re-weighted by the momentum spectrum of the corresponding run period it is being simulated for. This approximation allows us to estimate the contamination due to photons from MC with a reasonable assumption of their spectrum.






%%%%%%%%%%%%%%%%%%%%%%%%%%%%%%%%%
%% SECTION 4
%%%%%%%%%%%%%%%%%%%%%%%%%%%%%%%%%
%%\section{\textcolor{blue}{R\&D Strategy}}
%\include{section4}
%\input{section4}

%%%%%%%%%%%%%%%%%%%%%%%%%%%%%%%%%
%% SECTION 5
%%%%%%%%%%%%%%%%%%%%%%%%%%%%%%%%%
%\include{section4}
%\input{section5}

\newpage
\clearpage
\appendix
%\input{Appendix1}
\newpage
\clearpage
%\input{Appendix2}

\newpage
%%%%%%%%%%%%%%%%%%%%%%%%%%%%%%%%%
%%  BIBLIOGRAPHY	
%%%%%%%%%%%%%%%%%%%%%%%%%%%%%%%%%
%\include{reference}
\bibliographystyle{plain}
\bibliography{bib}

%\input{reference}


\end{document}